%! TEX root = ../rulebook.tex

\section{Committee Constitution}
\label{sec:Committee Constitution}

The following is the Constitution of the Major League Redditball Committee Room. The MLR Committee is committed to fighting for the players of Major League Redditball.

\subsection{ARTICLE I: ORGANIZATION}
\begin{deepEnumerate}
    \item The Committee consists of Players Association Representatives, General Managers, and members of the Office of the Commissioner.
\end{deepEnumerate}
    
\subsection{ARTICLE II: THE COMMITTEE}
\begin{deepEnumerate}
    \item The Committee shall be responsible for proposing, debating, and voting on changes to the MLR Rules.
    \item The Committee shall consist of at least one representative from each MLR Club. This person may be either the GM or a Player Representative.
    \begin{deepEnumerate}
        \item In order to be elected, Player Association Representatives must have been in the MLR for at least 3 sessions before they can be considered for the Committee Room.
        \begin{deepEnumerate}
            \item Before session 4 of each season, teams may appoint representatives who have less than 3 sessions experience as long as said representative 
            has been in the league since the beginning of said season.
            \item The committee may waive this rule for an incoming representative by majority (60\%) vote.
        \end{deepEnumerate}
    \end{deepEnumerate}
    \item Each Club will have ultimate authority over how their representatives are chosen and will have the freedom to recall and/or replace their representative at any time.
    \item Each Club must have one representative in the Committee Room at all times. Representatives may include both players or General Managers.
    \begin{deepEnumerate}
        \item The team will have a fourteen day grace period from the time of the loss of their voting member in the committee room to find a replacement.
        \item The team will have to keep a voting member for fourteen days to to be eligible for the fourteen day grace period.
    \end{deepEnumerate}
    \item In the event a person misses two consecutive votes in the committee room without prior discussion, they will be removed from the committee room 
    and ineligible to return to the committee room for twenty-one days.
\end{deepEnumerate}

\subsection{ARTICLE III: PROPOSALS}
\begin{deepEnumerate}
    \item Proposals may be put forth by any Committee member.
    \item There should only be one active proposal at any given time.
    \item Any proposal that would introduce a new rule to the playing Rulebook shall contain the text of the rule as it would be inserted into the Rulebook.
    \item Any proposal that alters an existing rule shall clearly indicate the changes as they would be displayed in the rulebook.
    \begin{deepEnumerate}
        \item If the proposal would change portions of a rule, it is suggested to indicate any removed portions using strikethrough.
    \end{deepEnumerate}
    \item If a proposal would remove a rule entirely, the proposal should indicate the rule or subrules to be removed. It need not contain the full text of those rules.
    \item If a Committee Room member wishes to propose a new rule or a rule change, the proposition may be seconded by one or more other members.
    This is however optional but highly encouraged.
    \begin{deepEnumerate}
        \item The sponsorship is meant to ensure that no committee room member has to defend a proposal by themselves, that any eventual proposal has (hopefully) already gotten 
        a few kinks worked out from constructive criticism from the sponsors before the vote even happens, and to prevent people from putting out a proposal on a whim.
    \end{deepEnumerate}
    \item When the proposal is ready to be put forth, a member of the OOTC must be notified and said member will ping @everyone to let the Committee Room know.
    \item Once a proposal is officially put forth, there will be a 24-hour discussion window. No voting will take place during this time.
    \begin{deepEnumerate}
        \item During that 24 hours, constructive discussion is encouraged, and the members who made the proposal will be encouraged to change the proposal based on concerns
        raised by other members. This is to ensure that no proposal ends up being a black/white issue, but instead that proposals are dynamic, and open to compromise 
        and change to make as many people as possible satisfied.
        \item A summary of what is being discussed by the committee will be placed on the committee bulletin board by a scribe as soon as possible 
        after the beginning of a discussion window.
    \end{deepEnumerate}
\end{deepEnumerate}

\subsection{ARTICLE IV: VOTING}
\begin{deepEnumerate}
    \item Following the 24 hour discussion period, where appropriate changes have been made to the proposal based on member suggestions, a vote may be called.
    \item Debate may be ended early or extended following an affirmative vote of 60\% of seated Committee members.
    \item Following the end of the debate, a member of the Office of the Commissioner will call for a vote using an @everyone ping and must include an attachment of the proposal.
    \begin{deepEnumerate}
        \item The vote will be pinned in the Committee Room.
        \item The vote will be acknowledged on the Committee bulletin board.
    \end{deepEnumerate}
    \item Official votes will begin between 11am-8pm EST, with the exception of federally recognized holidays.
    \begin{deepEnumerate}
        \item If a vote is missed due to an international holiday the person will not have the missed vote counted against them.
        \item Discussion may occur in the proposal discussion chats outside of these hours, but no official discussion period may begin outside them.
        \item Discussion outside these hours is not sanctioned as official discussion, and members who miss it should feel no obligation to catch themselves up on it.
    \end{deepEnumerate}
    \item Committee room members are required to vote on proposals within 24 hours, with the official time being the time of the @everyone ping calling for a vote.
    \begin{deepEnumerate}
        \item Failure to vote in 2 consecutive votes will result in removal from the Committee Room, including OOTC members.
        \begin{deepEnumerate}
            \item Committee members who will be unable to vote for a period of time must inform a member of the OOTC, and if necessary, a temporary representative will be chosen.
            \begin{deepEnumerate}
                \item Removal of these members will fall to the Community Member of the OOTC.
            \end{deepEnumerate}
            \item In cases of sudden emergencies the representative will be pardoned and if necessary a temporary representative will be chosen.
        \end{deepEnumerate}
    \end{deepEnumerate}
    \item The voting members of the Committee consist of up to 10 members of the OOTC and representatives of the MLR teams. The 10 OOTC members will be appointed 
    from the pool of 13 OOTC members by the League Operations Managers.
    \begin{deepEnumerate}
        \item The OOTC cannot represent a majority of the vote.
        \item Teams may have up to two representatives in the Committee Room (One GM, One Player Rep) but each team only has one vote.
        \begin{deepEnumerate}
            \item If there is a conflict between the Player Rep's and GM's votes, the GM's vote will be used.
        \end{deepEnumerate}
        \item The OOTC should be voting with the preservation and well being of the league in mind, and not necessarily their teams' interests.
        \item A vote will be considered valid only when 2/3 of Committee members vote, including voting to abstain. 
        \item A vote will be considered passed when 2/3 of the Committee members vote in favor of the proposal, excluding votes to abstain.
        A vote will be considered failed indefinitely when 51\% of the Committee Room votes against the proposal.
        \item If a vote gets 50\%-59\% approval, a continued discussion will take place and another vote will be put on hold for 48 hours.
        During this time adjustments to the proposals should be discussed.
        \item A proposal will only be voted on three times in a season before tabling until the next season.
        \item In the event that a vote reaches 40\% of the committee room abstaining, the proposal should be immediately brought back up for discussion.
    \end{deepEnumerate}
    \item As a result, the shortest amount of time that can possibly be between a proposal being put forth, and voting concluding, is 48 hours.
    \item In the event that a vote reaches a 60\% Majority no or abstaining the proposal will be tabled indefinitely.
    \item If there is a flaw found on one of the proposals that has already been moved to be voted on there will be a call for a 24-hour table to fix the issues found. 
    Then after the 24-hour table there will be a motion to revote on the proposal.
    \begin{deepEnumerate}
        \item A flaw will need to be noted specifically and supported by at least 3-5 other people.
        \item Once the vote is ready to recommence another @everyone ping will be issued.
    \end{deepEnumerate}
\end{deepEnumerate}

\subsection{ARTICLE V: EMERGENCY VOTES}
\begin{deepEnumerate}
    \item In the event that a situation arises that requires more immediate resolution an emergency vote may be called.
    \begin{deepEnumerate}
        \item An emergency vote should only be called to solve an existing problem, New rules cannot be passed via emergency vote.
        \item In the event an emergency vote is necessary and there is already a proposal either being discussed or voted on the current proposal will be temporarily
        tabled until the emergency vote is resolved, with the voting or discussion window being reset for the previous proposal.
        \item During an emergency vote, discussion may be brief, and OOTC may motion for a vote as soon as possible, at their discretion. 
        The voting period will be 12 hours, but may be shortened at OOTC discretion following a quorum of 60\% votes cast.
        \begin{deepEnumerate}
            \item There will be no punishment in the event an emergency vote is missed.
        \end{deepEnumerate}
        \item Emergency votes are only for  “This needs to be solved right now.” situations.
        \begin{deepEnumerate}
            \item The vote will pass after 12 hours with a 60\% majority of reporting voters or fail after 51\% of total voting members vote no.
            \item In the event that the vote passes with a 60\% reporting majority and not a total majority the solution may be brought up for revision after the session ends.
        \end{deepEnumerate}
        \item OOTC should be notified immediately if an emergency vote needs to happen as likely it will effect gameplay mechanics / teams in an immediate fashion.
    \end{deepEnumerate}
\end{deepEnumerate}

\subsection{ARTICLE VI: BREACH OF CONDUCT}
\begin{deepEnumerate}
    \item A representative may be recalled from office and replaced at any time by their Club.
    \item A representative may be permanently removed from the Committee if they display gross and repeated malfeasance, nonfeasance, or misfeasance, 
    (this decision will be reached by the Committee room as a whole - With the offender in question being removed.). Repeated incidents of incivility or immaturity
     will not be tolerated, OOTC members included, and appropriate warnings will be given by a member of the OOTC. 
     Continued failure to respond to said warnings may result in temporary or permanent removal.
     \item If any Committee member or members is found to be in breach of the rules of conduct, any member may bring a motion to remove said member in breach. 
     This motion takes precedence over any other business before the Committee, and MUST be seconded. The motion must include the specifics of the misconduct, 
     and prior attempts to remedy the behavior. The persons in violation will then have 24 hours to defend themselves prior to the committee’s vote on removal. 
     The vote to remove a Committee member must pass with at least 75\% of the vote.
     \begin{deepEnumerate}
         \item The user or users subject to removal proceedings will not be allowed to see the vote and will be quarantined from the committee room, 
         in order to preserve anonymity for those voting to remove.
     \end{deepEnumerate}
\end{deepEnumerate}

\subsection{ARTICLE VII: AMENDMENTS}
\begin{deepEnumerate}
    \item To make an amendment to the Constitution the same rules applied to Proposals will be followed.
\end{deepEnumerate}

\subsection{ARTICLE VIII: POST VOTE RULE MODIFICATIONS AND REPEALS}
\begin{deepEnumerate}
    \item To repeal/modify a rule there must be a proposal written and follow the same guidelines as if it were a proposal. 
    This proposal should state the changes that need to be made to this rule, including removal of said rule, and an explanation why you think the rule needs to be changed.
    \begin{deepEnumerate}
        \item This would require a 66\% affirmative votes to get a rule changed or removed.
        \item Rules that are in use and currently under emergency rule modification and/or repeal in active games should have the game put on hold 
        while a solution/vote is proposed.
    \end{deepEnumerate}
\end{deepEnumerate}

\subsection{ARTICLE IX: OPT-IN MEMBERS}
\begin{deepEnumerate}
    \item Members of the community may opt-in to committee proceedings at any time by notifying the OOTC of their request.
    \begin{deepEnumerate}
        \item The OOTC reserves the right to deny players access to the committee room for conflict of interest, or on grounds of previous misbehavior in the committee room.
        \item Opt-in members do not receive a vote, and will not be able to see vote totals.
        \item Opt-in members are held to the same discussion requirements as other committee members. They may be removed if they violate such requirements.
    \end{deepEnumerate}
\end{deepEnumerate}