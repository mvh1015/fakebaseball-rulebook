%! TEX root = ../rulebook.tex

\nonumsection{Committee Constitution}{sec:Committee Constitution}

The following is the Constitution of the Major League Redditball Committee. The MLR Committee is the body in charge of creating and passing rules for Major League Redditball.

\nonumsubsection{Article I: Committee Members}{sec:ARTICLE I}
\begin{deepEnumerate}
    \item The Committee is responsible for proposing, debating, and voting on changes to the MLR Rules.
    \item The Committee consists of up to two Representatives from each MLR Club and 10 OOTC Representatives. 
    \item The Representatives can be the GM and/or any players the GM designates.
    \begin{deepEnumerate}
        \item Representatives are designated by a team's GM at the GM's discretion. Each Club will have ultimate authority over how their Representatives are chosen
        and will have the freedom to recall and/or replace their Representatives at any time.
        \item Each Club must have at least one Representative in the Committee Room at all times.
        \item If a team has two Representatives, one must be designated the Primary Voter.
        \item The Representatives vote on behalf of the Club they represent.
    \end{deepEnumerate}
    \item Should an MLR Club have no Representatives, the GM shall act as the Representative until such time that a new Representative is designated,
    unless the GM is ineligible for Committee.
    \item The 10 OOTC members will be appointed from the pool of 13 OOTC members by the League Operations Managers.
    \begin{deepEnumerate}
        \item The OOTC members should represent the league and vote as stewards of the league, not their teams' interests.
    \end{deepEnumerate}
    \item Members of the community may request to join Committee proceedings as a neutral, non-voting Observer at any time by notifying a League Operations Manager of their request.
    \begin{deepEnumerate}
        \item The League Operations Managers reserve the right to deny players access to the committee room for conflict of interest
        or on grounds of previous misbehavior in the committee room.
        \item Observers do not receive a vote and will not be able to see vote totals.
    \end{deepEnumerate}
    \item The League Operations Managers can designate Scribes to assist in any matters related to the Committee.
    \begin{deepEnumerate}
        \item Scribes will have access to the Committee Room as necessary and may participate in discussions, but may not vote unless they are also a Representative.
        \item A Scribe shall be designated to edit and update the rulebook as rules are passed.
    \end{deepEnumerate}
\end{deepEnumerate}
    
\nonumsubsection{Article II: Submitting Proposals}{sec:ARTICLE II}
\begin{deepEnumerate}
    \item Any League member may submit a Proposal.
    \item Proposals may add, modify, or remove rule(s), including but not limited to the MLR Rulebook and any Appendices thereto.
    \item Only one proposal can be considered by the committee at any given time.
    \item Any proposal that would introduce a new rule to the playing Rulebook shall contain the text of the rule as it would be inserted into the Rulebook.
    \begin{deepEnumerate}
        \item If a rule is passed that does not contain the text of the rule as it should be inserted into the rulebook, the rulebook editor may reject it.
        If accepted, the rulebook editor shall codify it as close to the original intention as possible.
        \item The rulebook editor may make non-functional changes to a proposal, such as section numbers and references to other sections, as necessary to keep the rulebook organized.
    \end{deepEnumerate}
    \item If a proposal alters or removes an existing rule, the proposal shall clearly indicate which sections are being altered or removed.
    \begin{deepEnumerate}
        \item References to sections should use the sections in the most updated version of the Official Rulebook, found at \url{https://fakebaseball.gitlab.io/rulebook/rulebook.pdf}.
        \item Proposals should include the original rule language if it would assist in Committee Members understanding the proposal better.
    \end{deepEnumerate}
    \item If multiple proposals are submitted for discussion, Scribes will determine the order in which the proposals are discussed.
    \item Scribes may reject proposals due to being incomplete, failing to follow the rules, or other good cause.
\end{deepEnumerate}

\nonumsubsection{Article III: Discussing Proposals}{sec:ARTICLE III}
\begin{deepEnumerate}
    \item When a proposal is introduced for discussion, a Scribe must ping @everyone to let the Committee know.
    \begin{deepEnumerate}
        \item Proposals must be introduced on non-holiday weekdays between 11AM and 8PM EST.
        \item A Scribe should post the proposal on the Committee Bulletin Board after pinging the Committee.
    \end{deepEnumerate}
    \item There will be a 24 hour discussion window following the ping. No voting will take place during this time.
    \begin{deepEnumerate}
        \item During the 24 hour discussion window, constructive discussion is encouraged, and the member(s) who submitted the proposal may change the proposal
        based on concerns raised by other members.
        \item At any time during the 24 hour discussion window, the proposal may be withdrawn.
    \end{deepEnumerate}
    \item After the 24 hour discussion window, if the proposal has not been withdrawn, it will move to voting.
    \begin{deepEnumerate}
        \item A proposal cannot be modified once a vote has started.
        \item The 24 hour discussion window may be extended as necessary.
    \end{deepEnumerate}
\end{deepEnumerate}

\nonumsubsection{Article IV: Voting}{sec:ARTICLE IV}
\begin{deepEnumerate}
    \item Voting begins when a Scribe posts the proposal in Proposal Votes and pings @everyone that the vote is beginning.
    \item Voting must begin on non-holiday weekdays between 11AM and 8PM EST.
    \item The options for voting shall be Yes, No, and Abstain.
    \item Teams with two Representatives should coordinate to ensure that only one vote is cast on behalf of the team.
    \begin{deepEnumerate}
        \item If both Representatives of a team vote, only the vote of the Primary Voter will be counted.
    \end{deepEnumerate}
    \item A vote passes when at least 3/5ths of the Committee votes and at least 2/3rds of the votes are Yes.
    \begin{deepEnumerate}
        \item Abstain votes are counted for purposes of determining how much of the Committee has voted, but not for how many of the votes are Yes.
    \end{deepEnumerate}
    \item A Scribe shall post on the Committee Bulletin Board whether a vote has Passed or Failed.
\end{deepEnumerate}

\nonumsubsection{Article V: Rule Fixes}{sec:ARTICLE V}
\begin{deepEnumerate}
    \item If a small error is discovered after voting has begun or within 48 hours of the vote passing, any Committee Member may propose a Rule Fix.
    \item Rule Fixes should be submitted to a Scribe for review to ensure that it is a minor change directly related to a rule being voted on or recently passed.
    \item Rule Fixes follow the same rules as Proposals regarding discussion and voting, except:
    \begin{deepEnumerate}
        \item Rule Fixes shall have a 12 hour discussion period to ensure that the fix is needed and is written as intended.
        \item Rule Fixes shall have a 12 hour voting period.
        \item Rule Fixes require a 3/4ths majority of voting Members to vote Yes.
    \end{deepEnumerate}
\end{deepEnumerate}

\nonumsubsection{Article VI: Emergency Votes}{sec:ARTICLE VI}
\begin{deepEnumerate}
    \item In the event that a situation arises that requires more immediate resolution an emergency vote may be called.
    \item An emergency vote can only be called by any OOTC member to solve an existing problem which requires an immediate solution.
    \begin{deepEnumerate}
        \item Emergency votes are only to be used for “This absolutely needs to be solved right now.” situations.
    \end{deepEnumerate}
    \item In the event an emergency vote is called, the proposal being discussed or voted on, if any, will have its timer paused and all discussion
    halted until the emergency vote is resolved.
    \item Emergency votes are not required to follow the rules regarding timing or scheduling in Articles III and IV.
    \item The full discussion and voting period should be at least 12 hours if possible, but may be shortened if absolutely necessary.
    \item Emergency votes pass when 3/5ths of voting Committee Members vote Yes.
\end{deepEnumerate}

\nonumsubsection{Article VII: Breach of Conduct}{sec:ARTICLE VII}
\begin{deepEnumerate}
    \item A Committee Member may be permanently removed from the Committee for egregious or repeated severe improper conduct.
    \item Any Committee Member may bring a motion to remove said member in breach. 
    \begin{deepEnumerate}
        \item This motion takes precedence over any other business before the Committee, and must be seconded.
        \item The motion must include the specifics of the improper conduct.
        \item If the allegations include repeated severe improper conduct, the motion must include the prior improper conduct, with a link
        to the offending messages or actions if possible, and any actions taken to make the Member aware that their conduct was improper at the previous time.
        \item The Member must be pinged and given a chance to defend themselves not to exceed 24 hours.
    \end{deepEnumerate}
    \item The vote to remove a Committee Member must pass with at least 3/4s of voting Members voting Yes.
    \item The Member subject to removal proceedings will not be allowed to see the vote and will be quarantined from Committee in order to preserve anonymity for those
    voting to remove.
\end{deepEnumerate}



\nonumsubsection{Article VIII: Post Vote Rule Modifications and Repeals}{sec:ARTICLE VIII}
\begin{deepEnumerate}
    \item To repeal/modify a rule there must be a proposal written and follow the same guidelines as if it were a proposal. 
    This proposal should state the changes that need to be made to this rule, including removal of said rule, and an explanation why you think the rule needs to be changed.
    \begin{deepEnumerate}
        \item This would require a 66\% affirmative votes to get a rule changed or removed.
        \item Rules that are in use and currently under emergency rule modification and/or repeal in active games should have the game put on hold 
        while a solution/vote is proposed.
    \end{deepEnumerate}
\end{deepEnumerate}

\nonumsubsection{Article IX: Opt-in Members}{sec:ARTICLE IX}
\begin{deepEnumerate}
    \item Members of the community may opt-in to committee proceedings at any time by notifying the OOTC of their request.
    \begin{deepEnumerate}
        \item The OOTC reserves the right to deny players access to the committee room for conflict of interest, or on grounds of previous misbehavior in the committee room.
        \item Opt-in members do not receive a vote, and will not be able to see vote totals.
        \item Opt-in members are held to the same discussion requirements as other committee members. They may be removed if they violate such requirements.
    \end{deepEnumerate}
\end{deepEnumerate}